\chapter{Capacitor Design}

\section{Parallel Plate Capacitor}

We have made a simple $200mm\times150mm\times5mm$ design on SolidWorks and 3D printed the model with 2 different settings. The first model was printed with 75\% infill and the second model was printed with 15\% infill. The 3D printed model \& design are shown in Figure \ref{fig:3d-printed-models}.

\begin{figure}[h]
    \centering
    \includegraphics[width=0.9\linewidth]{assets/parallel-plate-design.png}
    \caption{3D Printed Model \& Design}
    \label{fig:3d-printed-models}
\end{figure}

We have covered the 3D printed model with aluminum foil to get a parallel plate capacitor.

\newpage
\thispagestyle{plain}

\section{Cylindirical Capacitor}
For cylindirical capacitor, we get 2 steel pipe one with 30mm diameter and 22cm length and the other with 40mm diameter and 22cm length. We have designed a lock mechanism to connect the pipes and make them a cylindirical capacitor. The design is shown in Figure \ref{fig:cylindirical-design}.

\begin{figure}[h]
    \centering
    \includegraphics[width=0.9\linewidth]{assets/cylindirical-design.png}
    \caption{Cylindirical Capacitor Design}
    \label{fig:cylindirical-design}
\end{figure}

To change the capacitance of the capacitor, we have designed a sliding mechanism to change the distance between the plates.
