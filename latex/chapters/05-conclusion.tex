\chapter{Conclusion}

Through this project, we learned how capacitors work, how to design one and how to use one in a circuit. We also learned how to measure the capacitance of a capacitor without using a DMM. This project helped us to get hands-on experience with capacitors, oscilloscopes, signal generators and understand their behavior in circuits. The results that we obtained from the experiments were consistent with our theoretical predictions. 

\section{Key Takeaways}
\begin{itemize}
    \item Capacitors store energy in the form of an electric field.
    \item In order to find capacitance, we can use square waves and measure the time taken to charge and discharge the capacitor.
    \item Feeding a triangular wave to a capacitor makes a low-pass filter.
    \item Charge stored in a capacitor is directly proportional to the voltage across it and it does not depend on the resistance in the circuit.
\end{itemize}
